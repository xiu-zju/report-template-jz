\documentclass{ctexart}
\usepackage{geometry}
\usepackage{enumitem}
\usepackage{graphicx}
\usepackage{listings} % 引入宏包
\usepackage{xcolor}   % 引入颜色宏包,用于代码块着色
\usepackage{titlesec}  % 用于自定义标题格式

\geometry{a4paper, left=3cm, right=3cm, top=3cm, bottom=3cm}
% 自定义一级标题的格式,左对齐,中文数字编号
\titleformat{\section}
  {\normalfont\Large\bfseries}  % 标题的字体、大小和粗体
  {\chinese{section}、}         % 自动生成中文编号(如一、二、三)
  {0em}                        % 编号与标题文字之间的间距
  {}                           % 标题前缀


\begin{document}
\lstset{                    % 设置代码显示风格
    language=Verilog,        % 代码语言
    basicstyle=\ttfamily,   % 基本字体设置
    keywordstyle=\color{blue},  % 关键字颜色
    commentstyle=\color{green}, % 注释颜色
    stringstyle=\color{red},    % 字符串颜色
    showstringspaces=false, % 不显示空格符
}
\pagestyle{empty}
\begin{center}
    \includegraphics[width=0.6\textwidth]{./image.png}
\end{center}
\vspace{1.8cm}

\begin{center}
    \zihao{2} \bf{本科实验报告}
\end{center}

\vspace{2.2cm}

\begin{center}
    \zihao{-3}
    \begin{tabular}{rl}
    课程名称 & : \underline{\makebox[8cm]{计算机组成}} \\
    \vspace{0.15cm} & \\
    姓\hspace{2em}名 & : \underline{\makebox[8cm]{某某某}} \\
    \vspace{0.15cm} & \\
    学\hspace{2em}院 & : \underline{\makebox[8cm]{计算机学院}} \\
    \vspace{0.15cm} & \\
    系 & : \underline{\makebox[8cm]{计算机科学与技术系}} \\
    \vspace{0.15cm} & \\
    专\hspace{2em}业 & : \underline{\makebox[8cm]{计算机科学与技术}} \\
    \vspace{0.15cm} & \\
    学\hspace{2em}号 & : \underline{\makebox[8cm]{xxxxxxxxxx}} \\
    \vspace{0.15cm} & \\
    指导教师 & : \underline{\makebox[8cm]{某某}}
    \end{tabular}
\end{center}

\vspace{1.8cm}
\begin{center}
    \zihao{-3}
    2024 年 13 月 32 日
\end{center}
\vspace{1cm}


\newpage
\begin{center}
    \zihao{3} \bf 浙江大学实验报告
\end{center}
\vspace{1cm}
\zihao{-4}
\noindent 课程名称: \underline{\makebox[5cm]{计算机组成}} 
实验类型: \underline{\makebox[5cm]{FPGA实验}} \\[0.4cm] 
实验项目名称: \underline{\makebox[11.3cm]{实验项目名称}} \\[0.4cm] 
学生姓名: \underline{\makebox[2cm]{某某某}} 
专业: \underline{\makebox[4.6cm]{计算机科学与技术}} 
学号: \underline{\makebox[3cm]{xxxxxxxxxx}} \\[0.4cm]
同组学生姓名: \underline{\makebox[4.2cm]{无}} 
指导老师: \underline{\makebox[5cm]{某某}} \\[0.4cm]
实验地点: \underline{\makebox[5cm]{实验地点}} 
实验日期: \underline{\makebox[1.4cm]{2024}} 年 \underline{\makebox[0.9cm]{13}} 月 \underline{\makebox[0.9cm]{32}} 日

\vspace{1.5cm}

\section{实验目的和要求}

\section{实验内容和原理}

\section{主要仪器设备}

\section{操作方法与实验步骤}

\section{实验数据记录和处理}

\section{实验结果与分析}

\section{讨论、心得}

\enddocument